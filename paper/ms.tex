\documentclass[usenatbib]{mn2e}

\usepackage{amsmath,amssymb}
\usepackage{graphicx}

\title{Towards model-order reduction in structure-formation simulations}

\author[B\"achtold and Saha]{Tino Valentin B\"achtold and Prasenjit Saha \\
Physik-Institut, University of Zurich, Winterthurerstr~190, 8057 Zurich, Switzerland \\ }

\date{}

\begin{document}

\maketitle

\begin{abstract}
\end{abstract}

\begin{keywords}

\end{keywords}

\section{Introduction}
Even though there is a rapid speed-up in computer performance, many real-life systems face challenges in numerical simulations. One of these challenges is the computation of the quantum mechanical wave function. Even for rather simple atoms the simulation of the behaviour of its constituents gets unmanageable and there is a necessity for a reduced order model. Also, recent paper indicates that theoretical dark matter particles may underlie a macroscopic quantum mechanical description. 
As a simple but non-trivial example, we use the nonlinear
Schr\"odinger (NLS) equation, which has been studied by several groups as
model for structure formation\cite{1993ApJ...416L..71W} and apply the proper orthogonal decomposition (POD) and its use with the Galerkin-projection.\t
Although the POD is a widely used method in the description of fluid dynamics, [2] it seems to be that there are only two papers where the POD was applied to the Schr\"odinger equation [3] [4]. In this paper we are focusing on the NLS equation with a self induced Poisson potential solve it with the split step Fourier method (SSM) and approximate it finally via POD. 
\section{NLS and the SSM}
The time dependent non-linear Schrodinger (NLS) equation emerges as a model describing the dynamics of wave packets in fibre optics, in the Bose-Einstein condensate theory and may predict the behaviour of the theoretical dark matter particle, the axion. It can be formulated by following equation.


\begin{equation} \label{eq1}
\begin{split}
i\dfrac {\partial } {\partial t}\psi( \overrightarrow{x},t) =( -\dfrac {\nabla ^{2}} {2}+ \xi \rho) \psi( \overrightarrow {x},t)
\end{split}
\end{equation}
where \(\xi\) is a parameter describing if the equation is focusing (\(\xi\)\textless  0) or defocusing (\(\xi\)\textgreater0) and \(\rho\) is the solution of the Poisson equtaion:

\begin{equation} \label{eq2}
\nabla \rho =\left| \psi\left( \overrightarrow {x},t\right) \right| ^{2}
\end{equation}

An ellegant way of solving (\ref{eq1}) is through the Fourier split step method where the equation gets split into the two parts:

\begin{equation} \label{eq4}
i\dfrac {\partial } {\partial t}\psi( \overrightarrow{x},t) = \xi  \rho  \psi( \overrightarrow {x},t)
\end{equation}

\begin{equation} \label{eq5}
i\dfrac {\partial } {\partial t}\psi( \overrightarrow{x},t) = -\dfrac {\nabla ^{2}} {2} \psi( \overrightarrow {x},t)
\end{equation}

By introducing the Fouriertransform \(\widetilde{\psi} (\overrightarrow {k},t)\) of \(\psi (\overrightarrow {x},t)\) in \ref{eq5} 

\begin{equation} \label{eq6}
i\dfrac {\partial } {\partial t}\widetilde{\psi}( \overrightarrow{x},t) = \dfrac {k ^{2}} {2} \widetilde{\psi}( \overrightarrow {x},t)
\end{equation}

you end up with a solvable system of equation. For a detailed look on this method we are recommending [XX]

\section{POD}
Since the above method requires to compute the fourier transform and its inverse every time-step, the simulation gets quite expensive by increasing the spatial resolution. We are therefore trying to provide a brief introduction to the POD which reduces eq \ref{eq1} to an ordinary differential equation of the form 

\begin{equation}\label{eq7}
\dfrac {\partial } {\partial t} \overrightarrow{a} = G( \overrightarrow{a},V)
\end{equation} 

Where G is a function depending solely on \(\overrightarrow{a}\) and some external values denoted as V. The premise to achieve this reduction is the approximation of the wave function \(\psi( \overrightarrow {x},t)\) as a  sum

\begin{equation}\label{eq8}
\psi( \overrightarrow {x},t) = \sum_{i=0}^M \alpha_i(t) \phi_i(\overrightarrow{x}) 
\end{equation} 

There are many types of this factorization. One could choose for example the Legendre polynomials or the Fourier series where for each of these choices, both the time-dependent function \(\alpha_i(t)\) and the spatial dependent function \(\phi(\overrightarrow{x})\) would change. In this paper \(\phi(\overrightarrow{x})\)is denoted as orthonormal basis called the POD-Modes. The original concept of the POD goes back to Pearson (1901) with the principal component analysis. The POD is also known as the Karhunen-Loève decomposition or the singular value decomposition (SVD) (for a detailed discussion about the connection between the PCA, KLD and the SVD, see [5]). The main goal of the POD can be formulated as follows: \textit{to find a basis of a subspace that optimally describes a set of data in a least square sense}. 
This can be seen by considering the wavefunction  \(\psi( \overrightarrow {x},t)\) evaluated at N grid points and at M time steps. These state variables are inserted into a data matrix A where the element \(A_{ij}\) denotes the amplitude of the wave function at the ith time and the jth grid point.

\bibliographystyle{mn2e}

\def\aap{A\&A}
\def\araa{ARA\&A}
\def\apjl{APJL}
\def\mnras{MNRAS}
\def\nat{Nature}

\bibliography{many}


\end{document}

